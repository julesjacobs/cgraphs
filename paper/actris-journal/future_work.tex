\section{Conclusion and future work}

In this paper, we have given a comprehensive account of the \lname logic,
which incorporates a protocol mechanism based on session types into concurrent
separation logic to enable functional correctness proofs of programs that
combine message-passing with other programming and concurrency paradigms.
Considering the rich literature on session types and concurrent separation
logic, we expect there to be many promising directions for future work.

One of the most prominent extensions of binary session types is multi-party
session types \cite{honda-POPL2008}, often called choreographies, which
allow concise specifications of message transfers between more
than two parties.
It would be interesting to explore a multi-party version of \pname,
similar to the multi-party version of session logic by \citet{costea-APLAS2018},
to allow \lname to more readily verify programs that make use of multi-party communication.

In addition to safety (\ie session fidelity), conventional session type systems
guarantee properties like deadlock and resource-leak freedom.
Since \lname is an extension of concurrent separation logic that supports
reasoning about several concurrency primitives and not only message passing,
ensuring deadlock freedom is hard.
The only prior work in this direction that we are aware of is by
\citet{hamin-ECOOP2019} and \citet{cracium-ICECCS2015}, but it is not immediately obvious how to integrate that
with Iris or \lname.
Resource-leak freedom has been studied in Iron, an extension of Iris by
\citet{bizjak-PACMPL2019}, which makes it possible to prove resource-leak
freedom of non-structured fork-based concurrent programs.
It would be interesting to build \pname on top of Iron instead of Iris.
